\documentstyle[10pt]{article}
\setlength\parskip{2ex}
\setlength\parindent{0pt}
\newcommand\shrink[1]{\hbox to 0pt{\parbox[t]{18cm}{#1}\hss}}
\newcommand\display[1]{\hspace{2em}\shrink{#1}}
\newcommand\lx{{Alex}}
\newcommand\lex{Lex}
\begin{document}
\title{\lx: A \lex\ for Haskell Programmers}
\author{Chris Dornan}
\date{29th September 1997}
\raggedbottom
\maketitle

\section{Introduction}

The \lx\ package, like \lex, takes a description of tokens based on regular
expressions and generates a program module for scanning text efficiently.  The
difference is that \lx\ generates Haskell modules rather than C/Ratfor source
files.  Although \lx\ takes after \lex, it is intended for Haskell programmers
and so departs quite radically from Lex in some respects.

\begin{figure}
\shrink{
\mbox{\tt {\char'45}{\char'173}}\\
\mbox{\tt module\ Tokens\ where}\\
\mbox{\tt }\\[-4pt]
\mbox{\tt import\ Alex}\\
\mbox{\tt {\char'45}{\char'175}}\\
\mbox{\tt }\\[-4pt]
\mbox{\tt }\\[-4pt]
\mbox{\tt {\char'173}\ {\char'136}d\ =\ 0-9\ \ \ \ \ \ {\char'175}\ \ \ \ \ \ \ \ \ \ \ \ \ \ \ \ \ \ \ \ \ \ \ --\ digits}\\
\mbox{\tt {\char'173}\ {\char'136}l\ =\ [a-zA-Z]\ {\char'175}\ \ \ \ \ \ \ \ \ \ \ \ \ \ \ \ \ \ \ \ \ \ \ --\ alphabetic\ characters}\\
\mbox{\tt }\\[-4pt]
\mbox{\tt "tokens{\char'137}lx"/"tokens{\char'137}acts":-}\\
\mbox{\tt }\\[-4pt]
\mbox{\tt \ \ <>\ \ \ \ \ ::=\ \ {\char'136}w+\ \ \ \ \ \ \ \ \ \ \ \ \ \ \ \ \ \ \ \ \ \ \ --\ white\ space}\\
\mbox{\tt \ \ <>\ \ \ \ \ ::=\ \ {\char'136}-{\char'136}-.*\ \ \ \ \ \ \ \ \ \ \ \ \ \ \ \ \ \ \ \ --\ comments}\\
\mbox{\tt \ \ <let'>\ ::=\ \ let\ \ \ \ \ \ \ \ \ \ \ \ \ \ \ \ \ \ \ \ \ \ \ {\char'45}{\char'173}\ let'\ p\ s\ =\ Let\ p\ \ \ \ \ \ \ \ \ \ {\char'45}{\char'175}}\\
\mbox{\tt \ \ <in'>\ \ ::=\ \ in\ \ \ \ \ \ \ \ \ \ \ \ \ \ \ \ \ \ \ \ \ \ \ \ {\char'45}{\char'173}\ in'\ \ p\ s\ =\ In\ \ p\ \ \ \ \ \ \ \ \ \ {\char'45}{\char'175}}\\
\mbox{\tt \ \ <int>\ \ ::=\ \ {\char'136}d+\ \ \ \ \ \ \ \ \ \ \ \ \ \ \ \ \ \ \ \ \ \ \ {\char'45}{\char'173}\ int\ \ p\ s\ =\ Int\ p\ (read\ s)\ {\char'45}{\char'175}\ }\\
\mbox{\tt \ \ <sym>\ \ ::=\ \ `=+-*/()'\ \ \ \ \ \ \ \ \ \ \ \ \ \ \ \ \ {\char'45}{\char'173}\ sym\ \ p\ s\ =\ Sym\ p\ (head\ s)\ {\char'45}{\char'175}}\\
\mbox{\tt \ \ <var>\ \ ::=\ \ {\char'136}l[{\char'136}l{\char'136}d{\char'136}{\char'137}{\char'136}']*\ \ \ \ \ \ \ \ \ \ \ \ \ {\char'45}{\char'173}\ var\ \ p\ s\ =\ Var\ p\ s\ \ \ \ \ \ \ \ {\char'45}{\char'175}}\\
\mbox{\tt }\\[-4pt]
\mbox{\tt }\\[-4pt]
\mbox{\tt {\char'45}{\char'173}}\\
\mbox{\tt data\ Token\ =}\\
\mbox{\tt \ \ \ \ \ \ \ \ Let\ Posn\ \ \ \ \ \ \ \ \ \ \ \ \ \ \ \ |}\\
\mbox{\tt \ \ \ \ \ \ \ \ In\ \ Posn\ \ \ \ \ \ \ \ \ \ \ \ \ \ \ \ |}\\
\mbox{\tt \ \ \ \ \ \ \ \ Sym\ Posn\ Char\ \ \ \ \ \ \ \ \ \ \ |}\\
\mbox{\tt \ \ \ \ \ \ \ \ Var\ Posn\ String\ \ \ \ \ \ \ \ \ |}\\
\mbox{\tt \ \ \ \ \ \ \ \ Int\ Posn\ Int\ \ \ \ \ \ \ \ \ \ \ \ |}\\
\mbox{\tt \ \ \ \ \ \ \ \ Err\ Posn}\\
\mbox{\tt \ \ \ \ \ \ \ \ deriving\ (Eq,Text)}\\
\mbox{\tt }\\[-4pt]
\mbox{\tt }\\[-4pt]
\mbox{\tt tokens::\ String\ ->\ [Token]}\\
\mbox{\tt tokens\ inp\ =\ scan\ tokens{\char'137}scan\ inp}\\
\mbox{\tt }\\[-4pt]
\mbox{\tt tokens{\char'137}scan::\ Scan\ Token}\\
\mbox{\tt tokens{\char'137}scan\ =\ load{\char'137}scan\ (tokens{\char'137}acts,stop{\char'137}act)\ tokens{\char'137}lx}\\
\mbox{\tt \ \ \ \ \ \ \ \ where}\\
\mbox{\tt \ \ \ \ \ \ \ \ stop{\char'137}act\ p\ ""\ \ =\ []}\\
\mbox{\tt \ \ \ \ \ \ \ \ stop{\char'137}act\ p\ inp\ =\ [Err\ p]}\\
\mbox{\tt {\char'45}{\char'175}}
}
\caption{A simple \lx\ specification.}\label{fig-tokens}
\end{figure}

A sample specification is given in Figure~\ref{fig-tokens}.  The first few
lines between the \mbox{\tt {\char'45}{\char'173}} and \mbox{\tt {\char'45}{\char'175}} provide a code scrap (some inlined Haskell
code) to be placed in the output.  All such code fragments will be placed in
order at the head of the module with the \lx-generated tables appearing at the
end.

The next two lines define the \mbox{\tt {\char'136}d} and \mbox{\tt {\char'136}l} macros for use in the token
definitions.

The \mbox{\tt "tokens{\char'137}lx"/"tokens{\char'137}acts":-} line starts the definition of a scanner.  It
is generated in two parts: the main tables being placed in \mbox{\tt token{\char'137}lx} and the
actions for each token being bound to \mbox{\tt tokens{\char'137}acts}.

The scanner is specified as a series of token definitions where each token
specification takes the form of

\display{
\mbox{\tt <}{\it token-id}\mbox{\tt >\ ::=\ }{\it rexp}
}

If {\it token-id} is omitted then the token will be discarded (this is done for
the first two white-space and comment token definitions), otherwise a
corresponding action function for constructing the token must be given
somewhere in the script.  In this case the action functions are given in the
code scraps to the right of the token definition but they could be specified
anywhere.  Here \lx\ differs from \lex; while each action must be
named in a code scrap, the programmer has more flexibility in laying out the
script.  Like comments, code scraps may be placed anywhere in the module.

The action function for each token takes the text matched and its position and
generates a token suitable for the parser, of type \mbox{\tt Token} in this case.

The remaining lines define the \mbox{\tt Token} data type and the scanner in two parts,
\mbox{\tt tokens{\char'137}scan} which uses \mbox{\tt load{\char'137}scan} for amalgamating the token actions, stop
action (for stopping the scanner) and token specification tables into a \mbox{\tt Scan}
structure, and \mbox{\tt tokens}, the scanner, which simply passes the \mbox{\tt Scan} structure
to \mbox{\tt scan}.  \mbox{\tt load{\char'137}scan}, \mbox{\tt Scan} and \mbox{\tt scan} were imported from the \mbox{\tt Scan} module
that comes with the \lx\ distribution.

While delegating the task of assembling the scanner to the programmer may seem
a bit bothersome, the effort is rewarded with a flexible and modular scheme for
generating scanners.

With this specification in \mbox{\tt Tokens.x}, \lx\ can be used to generate
\mbox{\tt Tokens.hs}:

\display{
\mbox{\tt alex\ Tokens.x}
}

If the module needed to be placed in different file, \mbox{\tt tkns.hs} for example,
then a second file-name can be specified on the command line:

\display{
\mbox{\tt alex\ Tokens.x\ tkns.hs}
}

The resulting module is Haskell~1.2 and Haskell~1.3 compatible.  It can also be
readily used with a Happy parser, with the catch that the \mbox{\tt Err} token must be
declared.

If the script were written with literate conventions then the \mbox{\tt .lx} extension
would be used instead of \mbox{\tt .x}.  (Literate scripts will be described in
Section~\ref{sec-lexical-syntax} on lexical syntax.)


\section{Syntax}

The syntax in this section is described with an extended BNF in which optional
phrases are enclosed in square brackets ($[\dots]$) and repeated phrases in
braces ($\{\dots\}$).  The terminal symbols {\sf ide}, {\sf tkn}, {\sf ch},
{\sf ech}, {\sf cch} {\sf smac}, {\sf rmac} and {\sf quot} are are defined in
Section~\ref{sec-lexical-syntax} on lexical syntax.


\subsection{Scripts, Macros and Scanners}

\lx\ scripts contain a list of scanner specifications, optionally preceded by
some global macro definitions.  Each scanner consists of a header with the
Haskell identifiers to be bound in the output module, a list of macro
definitions and a list of token definitions.  Macros may also be specified on
the right-hand-side of token definitions.

\display{
\begin{flushleft}\it\begin{tabbing}
\hspace{0.5in}\=\hspace{3.0in}\=\kill
$\it alex$\>\makebox[3.5em]{$\rightarrow$}$\it \{\ macdef\ \}\ \{\ scanner\ \}$\\ 
$\it $\\ 
$\it macdef$\>\makebox[3.5em]{$\rightarrow$}$\it \makebox{\tt {\char'173}}\ {\sf\ smac}\ \makebox{\tt =}\ set\ |\ {\sf\ rmac}\ \makebox{\tt =}\ rexp\ \makebox{\tt {\char'175}}$\\ 
$\it $\\ 
$\it scanner$\>\makebox[3.5em]{$\rightarrow$}$\it {\sf\ ide}\ [\makebox{\tt /}\ {\sf\ ide}]\ \makebox{\tt :-}\ \{\ macdef\ \}\ \{\ def\ \}$\\ 
$\it def$\>\makebox[3.5em]{$\rightarrow$}$\it {\sf\ tkn}\ \makebox{\tt ::=}\ \{\ macdef\ \}\ ctx$\\ 
$\it ctx$\>\makebox[3.5em]{$\rightarrow$}$\it \{\ sc\ \makebox{\tt :}\ \}\ [set\ \makebox{\tt {\char'134}}]\ rexp\ [\makebox{\tt /}\ rexp]$\\ 
$\it sc$\>\makebox[3.5em]{$\rightarrow$}$\it \makebox{\tt "0"}\ |\ {\sf\ ide}$
\end{tabbing}\end{flushleft}
}

Macros come in two flavours: regular expression macros and character set
macros.  A regular expression is more powerful than a character set but it can
be used in less contexts.  Regular expressions and character sets are described
below.

Macros obey a static scoping discipline with each macro scoping over the
construction it precedes.  Thus macros at the head of the script scope over the
whole script, macros preceding a scanner scope over the scanner and those on
the right-hand-side of a token definition will only be effective for the token
definition.  Because they are statically scoped any macros mentioned in the
body of a macro must be defined in its defining environment and its meaning is
then fixed at the point of definition.  For example, the definitions

\display{
\mbox{\tt {\char'173}\ {\char'136}g\ =\ {\char'136}a\ \ \ \ \ \ \ {\char'175}}\\
\mbox{\tt {\char'173}\ {\char'136}a\ =\ [a-zA-Z]\ {\char'175}}
}

will bind \mbox{\tt {\char'136}g} to the contents of the \mbox{\tt {\char'136}a} macro (which must be defined) and
rebind \mbox{\tt {\char'136}a} to match the alphabetic characters.  After both definitions, \mbox{\tt {\char'136}g}
will be bound to the original value of the \mbox{\tt {\char'136}a} macro.

The header line of a macro will usually give two identifiers: the first one for
binding the tables containing the token specifications and the second for
binding the action list.  If the second identifier is omitted then the action
list will not be generated and the programmer need not specify action functions
for each named token; however, to get a useful scanner with \mbox{\tt load{\char'137}scan}, a
hand-built action list will have to be supplied.

Leading context, trailing context and start codes may be specified with the
\mbox{\tt :}, \mbox{\tt {\char'134}}, \mbox{\tt /} operators.  These features will be described in
Section~\ref{sec-context-specifications} on context specifications.

A token identifier may be bound to more than one specification with the result
that all the definitions will be handled by the same action function.


\subsection{Regular Expressions}

\display{
\begin{flushleft}\it\begin{tabbing}
\hspace{0.5in}\=\hspace{3.0in}\=\kill
$\it rexp$\>\makebox[3.5em]{$\rightarrow$}$\it rexp_2\ \{\ \makebox{\tt |}\ rexp_2\ \}$\\ 
$\it rexp_2$\>\makebox[3.5em]{$\rightarrow$}$\it rexp_1\ \{\ rexp_1\ \}$\\ 
$\it rexp_1$\>\makebox[3.5em]{$\rightarrow$}$\it rexp_0\ [\ \makebox{\tt *}\ |\ \makebox{\tt +}\ |\ \makebox{\tt ?}\ |\ repeat\ ]$\\ 
$\it repeat$\>\makebox[3.5em]{$\rightarrow$}$\it \makebox{\tt {\char'173}}\ digit\ [\makebox{\tt ,}\ [digit]\ ]\ \makebox{\tt {\char'175}}$\\ 
$\it rexp_0$\>\makebox[3.5em]{$\rightarrow$}$\it \makebox{\tt {\char'44}}\ |\ {\sf\ rmac}\ |\ set\ |\ \makebox{\tt (}\ rexp\ \makebox{\tt )}$\\ 
$\it $\\ 
$\it digit$\>\makebox[3.5em]{$\rightarrow$}$\it \makebox{\tt 0}\ |\ \makebox{\tt 1}\ |\ \makebox{\tt 2}\ |\ \makebox{\tt 3}\ |\ \makebox{\tt 4}$\\ 
$\it $\>\makebox[3.5em]{$|$}$\it \makebox{\tt 5}\ |\ \makebox{\tt 6}\ |\ \makebox{\tt 7}\ |\ \makebox{\tt 8}\ |\ \makebox{\tt 9}$
\end{tabbing}\end{flushleft}
}

The regular expression syntax is similar to that of \lex, with the addition of
\mbox{\tt {\char'44}} for $\epsilon$, matching the empty string, and the \mbox{\tt {\char'45}}$\langle {\it
letter}\rangle$ syntax for regular-expression macros.

Here are some of the ways of repeating \mbox{\tt a}s.

\display{
\mbox{\tt "example{\char'137}rexps":-}\\
\mbox{\tt }\\[-4pt]
\mbox{\tt \ \ <a{\char'137}star>\ \ ::=\ {\char'44}\ |\ a+\ \ \ \ \ \ \ \ \ \ --\ =\ a*,\ zero\ or\ more\ as}\\
\mbox{\tt \ \ <a{\char'137}plus>\ \ ::=\ aa*\ \ \ \ \ \ \ \ \ \ \ \ \ --\ =\ a+,\ one\ or\ more\ as}\\
\mbox{\tt \ \ <a{\char'137}quest>\ ::=\ {\char'44}\ |\ a\ \ \ \ \ \ \ \ \ \ \ --\ =\ a?,\ zero\ or\ one\ as}\\
\mbox{\tt \ \ <a{\char'137}3>\ \ \ \ \ ::=\ a{\char'173}3{\char'175}\ \ \ \ \ \ \ \ \ \ \ \ --\ =\ aaa}\\
\mbox{\tt \ \ <a{\char'137}3{\char'137}5>\ \ \ ::=\ a{\char'173}3,5{\char'175}\ \ \ \ \ \ \ \ \ \ --\ =\ a{\char'173}3{\char'175}a?a?}\\
\mbox{\tt \ \ <a{\char'137}3{\char'137}>\ \ \ \ ::=\ a{\char'173}3,{\char'175}\ \ \ \ \ \ \ \ \ \ \ --\ =\ a{\char'173}3{\char'175}a*}
}


\subsection{Sets of Characters}

A set is a special form of regular expression that matches strings of length
one.  Here \lx\ differs markedly from \lex.

\display{
\begin{flushleft}\it\begin{tabbing}
\hspace{0.5in}\=\hspace{3.0in}\=\kill
$\it set$\>\makebox[3.5em]{$\rightarrow$}$\it set_0\ [\ \makebox{\tt {\char'43}}\ set_0\ ]$\\ 
$\it set_0$\>\makebox[3.5em]{$\rightarrow$}$\it \makebox{\tt {\char'176}}\ set_0$\\ 
$\it $\>\makebox[3.5em]{$|$}$\it chr\ [\ \makebox{\tt -}\ chr\ ]$\\ 
$\it $\>\makebox[3.5em]{$|$}$\it {\sf\ smac}$\\ 
$\it $\>\makebox[3.5em]{$|$}$\it \makebox{\tt [}\ \{set\}\ \makebox{\tt ]}$\\ 
$\it $\>\makebox[3.5em]{$|$}$\it {\sf\ quot}$\\ 
$\it chr$\>\makebox[3.5em]{$\rightarrow$}$\it {\sf\ ch}\ |\ {\sf\ ech}\ |\ {\sf\ cch}$
\end{tabbing}\end{flushleft}
}

The simplest set, {\it chr}, contains a single character.  The letters and
digits represent themselves while symbolic characters can be escaped with a
\mbox{\tt {\char'136}}.  Any character can be generated with a \mbox{\tt {\char'136}} followed by its decimal code,
though this is not portable.

A range of characters can be expressed by separating the characters with a
\mbox{\tt -}; all the characters with codes in the given range are included in
the set.  Character ranges can also be non-portable.

The union of a number of sets may be taken by enumerating them in square
brackets (\mbox{\tt [}\dots\mbox{\tt ]}), the complement of a set can be taken with \mbox{\tt {\char'176}} and the
difference of two sets can be taken with the \mbox{\tt {\char'43}} operator.  Finding a good use
for \mbox{\tt []} is left as an exercise for the devious reader.

A quoted set of characters can be expressed by enclosing it in quotes
(\mbox{\tt `}\dots\mbox{\tt '}).  Note that the quoted set starts with a back-quote and finishes
with a single-quote.  A \mbox{\tt '} character may not be included in such sets.

A set macro is expressed by a \mbox{\tt .} or by a \mbox{\tt {\char'136}} followed by a letter.  The
standard macros are listed in Figure~3.  Most of them consist of bindings for
the Haskell \mbox{\tt {\char'134}}$\langle letter \rangle$ character escape codes.  The \mbox{\tt {\char'136}w} and
\mbox{\tt {\char'136}p} correspond to the prelude \mbox{\tt isSpace} and \mbox{\tt isPrint} prelude functions.
\begin{figure} \mbox{\tt }\\
\mbox{\tt \ \ \ \ \ \ \ \ {\char'173}\ {\char'136}a\ =\ {\char'136}7\ \ \ \ \ \ \ \ \ \ \ \ \ {\char'175}\ \ \ --\ alarm}\\
\mbox{\tt \ \ \ \ \ \ \ \ {\char'173}\ {\char'136}b\ =\ {\char'136}8\ \ \ \ \ \ \ \ \ \ \ \ \ {\char'175}\ \ \ --\ back\ space}\\
\mbox{\tt \ \ \ \ \ \ \ \ {\char'173}\ {\char'136}t\ =\ {\char'136}9\ \ \ \ \ \ \ \ \ \ \ \ \ {\char'175}\ \ \ --\ form\ feed}\\
\mbox{\tt \ \ \ \ \ \ \ \ {\char'173}\ {\char'136}n\ =\ {\char'136}10\ \ \ \ \ \ \ \ \ \ \ \ {\char'175}\ \ \ --\ newline}\\
\mbox{\tt \ \ \ \ \ \ \ \ {\char'173}\ {\char'136}v\ =\ {\char'136}11\ \ \ \ \ \ \ \ \ \ \ \ {\char'175}\ \ \ --\ vertical\ tab}\\
\mbox{\tt \ \ \ \ \ \ \ \ {\char'173}\ {\char'136}f\ =\ {\char'136}12\ \ \ \ \ \ \ \ \ \ \ \ {\char'175}\ \ \ --\ form\ feed}\\
\mbox{\tt \ \ \ \ \ \ \ \ {\char'173}\ {\char'136}r\ =\ {\char'136}13\ \ \ \ \ \ \ \ \ \ \ \ {\char'175}\ \ \ --\ carriage\ return}\\
\mbox{\tt \ \ \ \ \ \ \ \ {\char'173}\ {\char'136}w\ =\ [{\char'136}t{\char'136}n{\char'136}v{\char'136}f{\char'136}r{\char'136}\ ]\ {\char'175}\ \ \ --\ white\ space}\\
\mbox{\tt \ \ \ \ \ \ \ \ {\char'173}\ {\char'136}p\ =\ {\char'136}32-{\char'136}126\ \ \ \ \ \ \ {\char'175}\ \ \ --\ printable/graphic\ characters}\\
\mbox{\tt \ \ \ \ \ \ \ \ {\char'173}\ .\ \ =\ {\char'136}0-{\char'136}255\ {\char'43}\ {\char'136}n\ \ \ {\char'175}\ \ \ --\ non-newline\ characters}
\caption{The standard macros (for Unix sytems).}\label{fig-standard-macros}
\end{figure}

\display{
\mbox{\tt "example{\char'137}sets":-}\\
\mbox{\tt }\\[-4pt]
\mbox{\tt \ \ <lls>\ \ \ \ \ \ ::=\ a-z\ \ \ \ \ \ \ \ \ \ \ \ \ \ \ --\ little\ letters}\\
\mbox{\tt \ \ <not{\char'137}lls>\ \ ::=\ {\char'176}a-z\ \ \ \ \ \ \ \ \ \ \ \ \ \ --\ anything\ but\ little\ letters}\\
\mbox{\tt \ \ <ls{\char'137}ds>\ \ \ \ ::=\ [a-zA-Z0-9]\ \ \ \ \ \ \ --\ letters\ and\ digits}\\
\mbox{\tt \ \ <sym>\ \ \ \ \ \ ::=\ `!@{\char'43}{\char'44}'\ \ \ \ \ \ \ \ \ \ \ \ --\ the\ symbols\ !,\ @,\ {\char'43}\ and\ {\char'44}}\\
\mbox{\tt \ \ <sym{\char'137}q{\char'137}nl>\ ::=\ [`!{\char'43}@{\char'44}'{\char'136}'{\char'136}n]\ \ \ \ \ \ --\ the\ above\ symbols\ with\ '\ and\ newline}\\
\mbox{\tt \ \ <quotable>\ ::=\ {\char'136}p{\char'43}{\char'136}'\ \ \ \ \ \ \ \ \ \ \ \ \ --\ any\ graphic\ character\ except\ '}\\
\mbox{\tt \ \ <del>\ \ \ \ \ \ ::=\ {\char'136}127\ \ \ \ \ \ \ \ \ \ \ \ \ \ --\ ASCII\ DEL}
}


\subsection{The Scanner's Alphabet}

The characters accepted by a scanner are precisely those permitted by the
regular expressions and the macro definitions.  This is usually determined by
the settings of the \mbox{\tt .} macro.  By default, \mbox{\tt .} includes every character except
newline but it could be redefined, for example, to exclude all the eight bit
codes:

\display{
\mbox{\tt {\char'173}\ .\ =\ {\char'136}0-{\char'136}127\ {\char'43}\ {\char'136}n\ {\char'175}}
}

This works because set complement, \mbox{\tt {\char'176}}$\langle{\it set}\rangle$ is defined to
mean \mbox{\tt [.{\char'136}n]{\char'43}}$\langle{\it set}\rangle$ so complemented sets would also exclude
eight-bit characters.

Note that the above restriction of the \mbox{\tt .} macro is unlikely to reduce the size
of the tables used by \lx\ or speed up the scanners generated by it, in fact,
the contrary, as the table formats were designed with the default configuration
in mind.


\subsection{Lexical Syntax}
\label{sec-lexical-syntax}

\lx\ supports the Haskell literate script convention as described in the
Haskell report (version 1.2).  See Figure~\ref{fig-lit} for an example literate
script.  If the name of the file containing the scripts ends in \mbox{\tt .lx} then the
lines that make up the script start with a \mbox{\tt >}; the script is preprocessed by
stripping out all the other lines and replacing the initial \mbox{\tt >} at the start of
each line with a space.  This process is formalised in
Section~\ref{sec-literate-scripts} where the scanner used to preprocess \lx\
scripts is given.

All white space appearing in the script is ignored, except where a space is
quoted with \mbox{\tt {\char'136}} or \mbox{\tt `}\dots\mbox{\tt '}.  Haskell-style line comments are introduced
with \mbox{\tt --} and code scraps are enclosed in \mbox{\tt {\char'45}{\char'173}} \dots \mbox{\tt {\char'45}{\char'175}} brackets.  Otherwise,

\display{
\begin{tabular}{ll}
{\sf ide}       &is a Haskell identifier in quotes (\mbox{\tt "}\dots\mbox{\tt "}). \\
{\sf tkn}       &is an optional Haskell identifier in angle
                                                brackets (\mbox{\tt <}\dots\mbox{\tt >}). \\
{\sf ch}        &is a letter or digit. \\
{\sf ech}       &is a \mbox{\tt {\char'136}} followed by a symbolic character. \\
{\sf cch}       &is a \mbox{\tt {\char'136}} followed by a character code. \\
{\sf smac}      &is either a \mbox{\tt .} or a \mbox{\tt {\char'136}} followed by a letter. \\
{\sf rmac}      &is a \mbox{\tt {\char'45}} followed by a letter. \\
{\sf quot}      &is a sequence of non-\mbox{\tt '} characters in quotes
                                                        (\mbox{\tt `}\dots\mbox{\tt '}). \\
\end{tabular}
}

The lexical syntax is formalised by the \lx\ script in Figure~\ref{fig-lx}.
\begin{figure}
\shrink{
\mbox{\tt {\char'173}\ {\char'136}s\ =\ {\char'136}w{\char'43}{\char'136}n\ \ \ \ \ \ {\char'175}\ \ \ \ \ \ \ \ \ \ \ \ \ \ \ \ \ \ \ \ \ \ \ \ \ \ \ \ \ \ \ \ \ \ \ \ \ --\ spaces\ +\ tabs,\ etc}\\
\mbox{\tt {\char'173}\ {\char'136}d\ =\ 0-9\ \ \ \ \ \ \ \ {\char'175}\ \ \ \ \ \ \ \ \ \ \ \ \ \ \ \ \ \ \ \ \ \ \ \ \ \ \ \ \ \ \ \ \ \ \ \ \ --\ digits}\\
\mbox{\tt {\char'173}\ {\char'136}a\ =\ a-z\ \ \ \ \ \ \ \ {\char'175}\ \ \ \ \ \ \ \ \ \ \ \ \ \ \ \ \ \ \ \ \ \ \ \ \ \ \ \ \ \ \ \ \ \ \ \ \ --\ lower-case\ alphas}\\
\mbox{\tt {\char'173}\ {\char'136}A\ =\ A-Z\ \ \ \ \ \ \ \ {\char'175}\ \ \ \ \ \ \ \ \ \ \ \ \ \ \ \ \ \ \ \ \ \ \ \ \ \ \ \ \ \ \ \ \ \ \ \ \ --\ upper-case\ alphas}\\
\mbox{\tt {\char'173}\ {\char'136}l\ =\ [{\char'136}a{\char'136}A]\ \ \ \ \ {\char'175}\ \ \ \ \ \ \ \ \ \ \ \ \ \ \ \ \ \ \ \ \ \ \ \ \ \ \ \ \ \ \ \ \ \ \ \ \ --\ alpha\ characters}\\
\mbox{\tt {\char'173}\ {\char'136}i\ =\ [{\char'136}l{\char'136}d{\char'136}{\char'137}{\char'136}']\ {\char'175}\ \ \ \ \ \ \ \ \ \ \ \ \ \ \ \ \ \ \ \ \ \ \ \ \ \ \ \ \ \ \ \ \ \ \ \ \ --\ identifier\ trailer}\\
\mbox{\tt }\\[-4pt]
\mbox{\tt "alex{\char'137}lx"\ :-}\\
\mbox{\tt }\\[-4pt]
\mbox{\tt \ \ <>\ \ \ \ \ ::=\ \ {\char'136}w+\ \ \ \ \ \ \ \ \ \ \ \ \ \ \ \ \ \ \ \ \ \ \ \ \ \ \ \ \ \ \ \ \ \ \ \ \ \ \ --\ white\ space}\\
\mbox{\tt \ \ <>\ \ \ \ \ ::=\ \ {\char'136}-{\char'136}-.*\ \ \ \ \ \ \ \ \ \ \ \ \ \ \ \ \ \ \ \ \ \ \ \ \ \ \ \ \ \ \ \ \ \ \ \ --\ comments}\\
\mbox{\tt \ \ <code>\ ::=\ \ \ \ \ \ \ \ \ \ \ \ \ \ \ \ \ \ \ \ \ \ \ \ \ \ \ \ \ \ \ \ \ \ \ \ \ \ \ \ \ \ \ \ --\ code\ scraps:}\\
\mbox{\tt \ \ \ \ \ \ \ \ {\char'136}{\char'45}{\char'136}{\char'173}\ (.{\char'43}{\char'136}{\char'45}|{\char'136}{\char'45}.{\char'43}{\char'136}{\char'175})*\ {\char'136}{\char'45}{\char'136}{\char'175}\ \ \ \ \ \ \ \ \ \ \ \ \ \ \ \ \ \ \ \ \ \ \ \ --\ \ \ single-line\ scraps}\\
\mbox{\tt \ \ \ \ \ \ \ \ |\ \ {\char'136}{\char'45}{\char'136}{\char'173}{\char'136}s*{\char'136}n\ (((.{\char'43}{\char'136}{\char'45}.*)?|{\char'136}{\char'45}(.{\char'43}{\char'136}{\char'175}.*)?){\char'136}n)*\ {\char'136}{\char'45}{\char'136}{\char'175}\ \ --\ \ \ multi-line\ scraps}\\
\mbox{\tt \ \ \ \ \ \ \ \ |\ \ {\char'136}{\char'45}{\char'136}{\char'173}{\char'136}s*{\char'136}n\ \ \ \ \ \ \ \ \ \ \ \ \ \ \ \ \ \ \ \ \ \ \ \ \ \ \ \ \ \ \ \ \ \ \ \ --\ \ \ multi-line\ scraps}\\
\mbox{\tt \ \ \ \ \ \ \ \ \ \ \ \ \ \ (((.{\char'43}{\char'136}\ .*)?|{\char'136}\ (.{\char'43}{\char'136}{\char'45}.*)?|{\char'136}\ {\char'136}{\char'45}(.{\char'43}{\char'136}{\char'175}.*)?){\char'136}n)*--\ \ \ \ \ \ (literate}\\
\mbox{\tt \ \ \ \ \ \ \ \ \ \ \ \ \ \ \ \ \ \ \ \ \ \ \ \ {\char'136}\ {\char'136}{\char'45}{\char'136}{\char'175}\ \ \ \ \ \ \ \ \ \ \ \ \ \ \ \ \ \ \ \ \ \ \ \ \ \ --\ \ \ \ \ \ \ scripts)}\\
\mbox{\tt \ \ <zero>\ ::=\ \ {\char'136}"\ 0\ {\char'136}"\ \ \ \ \ \ \ \ \ \ \ \ \ \ \ \ \ \ \ \ \ \ \ \ \ \ \ \ \ \ \ \ \ \ \ --\ "0"\ start\ code}\\
\mbox{\tt \ \ <ide>\ \ ::=\ \ {\char'136}"\ {\char'136}a{\char'136}i*\ {\char'136}"\ \ \ \ \ \ \ \ \ \ \ \ \ \ \ \ \ \ \ \ \ \ \ \ \ \ \ \ \ \ \ --\ function\ identifier}\\
\mbox{\tt \ \ <tkn>\ \ ::=\ \ {\char'136}<\ ({\char'136}a{\char'136}i*)?\ {\char'136}>\ \ \ \ \ \ \ \ \ \ \ \ \ \ \ \ \ \ \ \ \ \ \ \ \ \ \ \ --\ token\ identifier}\\
\mbox{\tt \ \ <bnd>\ \ ::=\ \ {\char'136}:{\char'136}-\ \ \ \ \ \ \ \ \ \ \ \ \ \ \ \ \ \ \ \ \ \ \ \ \ \ \ \ \ \ \ \ \ \ \ \ \ \ --\ ":-"}\\
\mbox{\tt \ \ <prd>\ \ ::=\ \ {\char'136}:{\char'136}:{\char'136}=\ \ \ \ \ \ \ \ \ \ \ \ \ \ \ \ \ \ \ \ \ \ \ \ \ \ \ \ \ \ \ \ \ \ \ \ --\ "::="}\\
\mbox{\tt \ \ <spe>\ \ ::=\ \ `{\char'173}={\char'175}:{\char'134}/|*+?,{\char'44}(){\char'43}[]-'\ \ \ \ \ \ \ \ \ \ \ \ \ \ \ \ \ \ \ \ \ \ --\ specials}\\
\mbox{\tt \ \ <ch>\ \ \ ::=\ \ [{\char'136}l{\char'136}d]\ \ \ \ \ \ \ \ \ \ \ \ \ \ \ \ \ \ \ \ \ \ \ \ \ \ \ \ \ \ \ \ \ \ \ \ --\ letter\ or\ digit}\\
\mbox{\tt \ \ <ech>\ \ ::=\ \ {\char'136}{\char'136}\ {\char'136}p{\char'43}[{\char'136}l{\char'136}d]\ \ \ \ \ \ \ \ \ \ \ \ \ \ \ \ \ \ \ \ \ \ \ \ \ \ \ \ \ \ --\ escaped\ symbols}\\
\mbox{\tt \ \ <cch>\ \ ::=\ \ {\char'136}{\char'136}\ {\char'136}d{\char'173}1,3{\char'175}\ \ \ \ \ \ \ \ \ \ \ \ \ \ \ \ \ \ \ \ \ \ \ \ \ \ \ \ \ \ \ \ --\ character\ codes}\\
\mbox{\tt \ \ <smac>\ ::=\ \ {\char'136}{\char'136}\ {\char'136}l\ |\ {\char'136}.\ \ \ \ \ \ \ \ \ \ \ \ \ \ \ \ \ \ \ \ \ \ \ \ \ \ \ \ \ \ \ \ --\ set\ macros}\\
\mbox{\tt \ \ <rmac>\ ::=\ \ {\char'136}{\char'45}\ {\char'136}l\ \ \ \ \ \ \ \ \ \ \ \ \ \ \ \ \ \ \ \ \ \ \ \ \ \ \ \ \ \ \ \ \ \ \ \ \ --\ rexp\ macros}\\
\mbox{\tt \ \ <quot>\ ::=\ \ {\char'136}`\ {\char'136}p{\char'43}{\char'136}'+\ {\char'136}'\ \ \ \ \ \ \ \ \ \ \ \ \ \ \ \ \ \ \ \ \ \ \ \ \ \ \ \ \ \ --\ quoted\ sets}
}
\caption{The \lx\ scanner.}\label{fig-lx}
\end{figure}

As can be seen from the \mbox{\tt <code>} token, code scraps come in two varieties:
those that start and end with a newline and those that are contained on one
line.  If the code scrap starts with a newline then it must finish with the
\mbox{\tt {\char'45}{\char'175}} at the start of the line.  (In fact, it may have a single space between
the newline and the \mbox{\tt {\char'45}{\char'175}}; this is to accomodate literate scripts where the \mbox{\tt >}
between the newline and the \mbox{\tt {\char'45}{\char'175}} is converted to a space.)  This means that the
\mbox{\tt {\char'45}{\char'175}} sequence may be used anywhere in the code scrap except at the start of the
line and that the text from the first newline to the last newline can be copied
without alteration into the output module.

The code in multi-line code scraps must follow the same layout conventions used
for the tables generated by \lx, namely that all top-level defintions start in
the left-hand column for ordinary scripts, column two for literate scripts.

Line code scraps may not contain the \mbox{\tt {\char'45}{\char'175}} sequence anywhere in them.  They will
appear in the output in the left hand column for ordinary scripts, at column
two for literate scripts.


\section{General Scanners}
\label{sec-general-scanners}

\subsection{The Alex Module}

The \mbox{\tt Alex} module contains the run-time interface.  It is self-contained so
only \mbox{\tt Alex} and the \lx\ generated modules need to be added to programs using a
\lx\ scanner.

The \mbox{\tt Posn} data type is the first product of the \mbox{\tt Alex} module.  It provides a
standard means of positioning tokens in the input stream.

\display{
\mbox{\tt data\ Posn\ =\ Pn\ Int\ Int\ Int\ \ deriving\ (Eq,Text)}\\
\mbox{\tt }\\[-4pt]
\mbox{\tt start{\char'137}pos::\ Posn}\\
\mbox{\tt start{\char'137}pos\ =\ Pn\ 0\ 1\ 1}\\
\mbox{\tt }\\[-4pt]
\mbox{\tt eof{\char'137}pos::\ Posn}\\
\mbox{\tt eof{\char'137}pos\ =\ Pn\ (-1)\ (-1)\ (-1)}
}

\mbox{\tt Pn\ addr\ ln\ col} represents the location of a token found \mbox{\tt addr} characters
into the file on line \mbox{\tt ln} and column \mbox{\tt col}.  In calculating the column
position, it will be assumed that tab characters use eight-character tab stops.
The first character of the file is located at \mbox{\tt start{\char'137}pos} and \mbox{\tt eof{\char'137}pos}, by
convention, will represent the end of file.

The \mbox{\tt Alex} module provides two packages for generating scanners from the tables
generated by \lx: the basic \mbox{\tt Scan}/\mbox{\tt load{\char'137}scan}/\mbox{\tt scan} package used for the
\mbox{\tt Token} module of Figure~\ref{fig-tokens}, and a more flexible
\mbox{\tt GScan}/\mbox{\tt load{\char'137}gscan}/\mbox{\tt gscan} package.

The \mbox{\tt scan} package generates simple scanners that convert input text to streams
of tokens.  The scanners are stateless as each token generated is a function of
its textual content and location.
 
The token actions take the form of an association list associating each token
name with an action function that constructs the token from the text matched
and its location.  The stop action is invoked when no more input can be
tokenised; it takes the residual input and its position and generates the
remaining stream of tokens, usually the empty list or an end-of-file token if
the empty string is passed, an error token otherwise.

\display{
\mbox{\tt type\ Actions\ t\ =\ ([(String,TokenAction\ t)],\ StopAction\ t)}\\
\mbox{\tt }\\[-4pt]
\mbox{\tt type\ TokenAction\ t\ =\ Posn\ ->\ String\ ->\ t}\\
\mbox{\tt }\\[-4pt]
\mbox{\tt type\ StopAction\ t\ =\ Posn\ ->\ String\ ->\ [t]}
}

\mbox{\tt load{\char'137}scan} combines the actions with the dump generated by \lx\ to produce a
\mbox{\tt Scan} structure that can be passed to \mbox{\tt scan}.  \mbox{\tt scan} takes the scanner and
the input text and generates a stream of tokens.  It assumes that the text is
at the start of the input with the position set to \mbox{\tt start{\char'137}pos} (see above) and
sets the last character read to newline (the last character read is used to
resolve leading context specifications); \mbox{\tt scan'} can be used to override these
defaults.

\display{
\mbox{\tt load{\char'137}scan::\ Actions\ t\ ->\ DFADump\ ->\ Scan\ t}\\
\mbox{\tt scan::\ Scan\ t\ ->\ String\ ->\ [t]}\\
\mbox{\tt scan'::\ Scan\ t\ ->\ Posn\ ->\ Char\ ->\ String\ ->\ [t]}
}

The \mbox{\tt gscan} package generates general-purpose scanners for converting input
text into a return type determined by the application.  Access to the scanner's
internal state, start codes and some application-specific state is provided.
 
The token actions take the form of an association list associating each token
name with an action function that constructs the result from the length of the
token, the scanner's state (including the remaining input from the start of the
token) and a continuation function that scans the remaining input.
 
More specifically, each token action takes as arguments the position of the
token, the last character read before the token (used to resolve leading
context), the whole input from the start of the token, the length of the token,
the continuation function and the visible state (as distinct from the scanner's
internal state) including the current start code and the application specific
state.  The stop action is invoked when no more input can be scanned; it takes
the same parameters as the token actions without the token length and the
continuation function.

\display{
\mbox{\tt type\ GScan\ s\ r\ =\ (DFA\ (GTokenAction\ s\ r),\ GStopAction\ s\ r)}\\
\mbox{\tt }\\[-4pt]
\mbox{\tt type\ GActions\ s\ r\ =\ ([(String,\ GTokenAction\ s\ r)],\ GStopAction\ s\ r)}\\
\mbox{\tt }\\[-4pt]
\mbox{\tt type\ GTokenAction\ s\ r\ =\ }\\
\mbox{\tt \ \ \ \ \ \ \ \ Posn\ ->\ Char\ ->\ String\ ->\ Int\ ->}\\
\mbox{\tt \ \ \ \ \ \ \ \ \ \ \ \ \ \ \ \ ((StartCode,s)->r)\ ->\ (StartCode,s)\ ->\ r}\\
\mbox{\tt }\\[-4pt]
\mbox{\tt type\ GStopAction\ s\ r\ =\ Posn\ ->\ Char\ ->\ String\ ->\ (StartCode,s)\ ->\ r}
}


\mbox{\tt load{\char'137}gscan} combines the actions with the dump generated by \lx\ to produce a
\mbox{\tt GScan} structure that can be passed to \mbox{\tt gscan}.  \mbox{\tt gscan} takes the scanner,
the application-specific state and the input text as parameters.  It assumes
that the text is at the start of the input with the position set to
\mbox{\tt start{\char'137}pos} (see above) and sets the last character read to newline and the
start code to 0; \mbox{\tt gscan'} can be used to override these defaults.

\display{
\mbox{\tt load{\char'137}gscan::\ GActions\ s\ r\ ->\ DFADump\ ->\ GScan\ s\ r}\\
\mbox{\tt gscan::\ GScan\ s\ r\ ->\ s\ ->\ String\ ->\ r}\\
\mbox{\tt gscan'::\ GScan\ s\ r\ ->\ Posn\ ->\ Char\ ->\ String\ ->\ (StartCode,s)\ ->\ r}
}

Note that a token action can ignore its continuation function and call up
\mbox{\tt gscan'} with a different scanner to tokenise the rest of the input.  This
offers a more efficient alternative to start codes (see
Section~\ref{sec-context-specifications} on context specifications) for
invoking alternate scanners on segments of the input.


\subsection{Stateful Scanners}
\label{sec-stateful-scanners}

Some parsing constructions are best handled by making the scanner stateful,
allowing the action functions to read and alter some state.  A stateful scanner
will instantiate the \mbox{\tt s} parameter of \mbox{\tt GScan} with the state type needed by the
application and will make use of the \mbox{\tt (StartCode,s)} argument of the action
functions.  (The \mbox{\tt StartCode} component of the scanner's state will be dealt
with in Section~\ref{sec-context-specifications} on context specifications.)

Consider the problem of collecting the code scraps from a \lx\ script.  The
code scraps could have been included in the grammar, forcing them to appear at
certain points in the script and complicating the grammar and parser.  Instead,
they are ignored by the parser and collected in the scanner's state, being
passed back to the parser in an explicit end-of-file token.

A simple scanner illustrating this technique is given in
Figure~\ref{fig-state}.  It returns a stream of identifiers terminated with an
end-of-file token containing all the accumulated code scraps on the input.

The idea for this technique, and another potential application of it, came from
the Brisk scanner which constructs the symbol table, returning integer handles
in the token stream and the symbol table in the end-of-file token.

\begin{figure}
\shrink{
\mbox{\tt {\char'45}{\char'173}}\\
\mbox{\tt import\ Alex}\\
\mbox{\tt {\char'45}{\char'175}}\\
\mbox{\tt }\\[-4pt]
\mbox{\tt "state{\char'137}lx"/"state{\char'137}acts":-}\\
\mbox{\tt }\\[-4pt]
\mbox{\tt \ \ <>\ \ \ \ \ ::=\ {\char'136}w+}\\
\mbox{\tt \ \ <code>\ ::=\ {\char'136}{\char'45}{\char'136}{\char'173}\ ({\char'176}{\char'136}{\char'45}\ |\ {\char'136}{\char'45}{\char'176}{\char'136}{\char'175})*\ {\char'136}{\char'45}{\char'136}{\char'175}}\\
\mbox{\tt \ \ <ide>\ \ ::=\ [A-Za-z]+}\\
\mbox{\tt }\\[-4pt]
\mbox{\tt }\\[-4pt]
\mbox{\tt {\char'45}{\char'173}}\\
\mbox{\tt code\ {\char'137}\ {\char'137}\ inp\ len\ cont\ (sc,frags)\ =\ cont\ (sc,frag:frags)}\\
\mbox{\tt \ \ \ \ \ \ \ \ where}\\
\mbox{\tt \ \ \ \ \ \ \ \ frag\ =\ take\ (len-4)\ (drop\ 2\ inp)}\\
\mbox{\tt }\\[-4pt]
\mbox{\tt ide\ {\char'137}\ {\char'137}\ inp\ len\ cont\ st\ =\ Ide\ (take\ len\ inp):cont\ st}\\
\mbox{\tt }\\[-4pt]
\mbox{\tt }\\[-4pt]
\mbox{\tt data\ Token\ =\ Ide\ String\ |\ Eof\ String\ |\ Err}\\
\mbox{\tt }\\[-4pt]
\mbox{\tt }\\[-4pt]
\mbox{\tt tokens::\ String\ ->\ [Token]}\\
\mbox{\tt tokens\ inp\ =\ gscan\ state{\char'137}scan\ []\ inp}\\
\mbox{\tt }\\[-4pt]
\mbox{\tt state{\char'137}scan::\ GScan\ [String]\ [Token]}\\
\mbox{\tt state{\char'137}scan\ =\ load{\char'137}gscan\ (state{\char'137}acts,stop{\char'137}act)\ state{\char'137}lx}\\
\mbox{\tt \ \ \ \ \ \ \ \ where}\\
\mbox{\tt \ \ \ \ \ \ \ \ stop{\char'137}act\ {\char'137}\ {\char'137}\ ""\ ({\char'137},frags)\ =\ [Eof\ (unlines(reverse\ frags))]}\\
\mbox{\tt \ \ \ \ \ \ \ \ stop{\char'137}act\ {\char'137}\ {\char'137}\ {\char'137}\ {\char'137}\ =\ [Err]}\\
\mbox{\tt {\char'45}{\char'175}}
}
\caption{A stateful scanner.}\label{fig-state}
\end{figure}


\subsection{Literate Scripts}
\label{sec-literate-scripts}

The \mbox{\tt scan} package only provides for actions that return a single token as part
of a list of such tokens.  Sometimes a more flexible format is required, such
as a preprocessor that generates another stream of characters.

\lx\ itself uses such a preprocessor to deal with literate scripts.  Recall
that each line in a (Haskell) literate scripts is either a blank line, a code
scrap line starting with a \mbox{\tt >} or a comment line, and that comment lines and
scrap lines must be separated by one or more blank lines.  The script in
Figure~\ref{fig-lit} defines three macros, \mbox{\tt {\char'45}b}, \mbox{\tt {\char'45}s} and \mbox{\tt {\char'45}c}, for recognising
blank, scrap and comment lines (where each line {\em starts} with a newline
character).  Figure~\ref{fig-lit} is itself a literate script, of course.

If two newline characters are added to the front of the input then a valid
literate script could be considered as a series of scraps and comments in which
a scrap consists of a blank line followed by one or more scrap lines and a
comment consists of a blank line followed by zero or more comment lines.  Note
that the initial lines in a series of blank lines will each be considered
comments in this scheme.

The action for the \mbox{\tt <scrap>} token replaces each of the \mbox{\tt >} in column one with
a space, appending the continuation onto the result.  The \mbox{\tt <comment>} action
strips out everything except the newlines, appending its continuation. (The
newlines from the comment scraps are retained in order to keep the line numbers
synchronised with the original input.)

To construct the \mbox{\tt literate} scanner, of type \mbox{\tt String\ ->\ String}, we must
remember to insert the dummy newlines onto the input, and to remove them again
afterwards.

\begin{figure}
\shrink{
\mbox{\tt >{\char'45}{\char'173}\ import\ Alex\ {\char'45}{\char'175}}\\
\mbox{\tt }\\[-4pt]
\mbox{\tt }\\[-4pt]
\mbox{\tt >\ "lit{\char'137}lx"/"lit{\char'137}acts":-}\\
\mbox{\tt }\\[-4pt]
\mbox{\tt >\ {\char'173}\ {\char'136}s\ =\ {\char'136}w{\char'43}{\char'136}n\ \ \ \ \ \ \ \ \ \ \ \ \ \ \ \ \ \ {\char'175}}\\
\mbox{\tt >\ {\char'173}\ {\char'45}b\ =\ {\char'136}n{\char'136}s*\ \ \ \ \ \ \ \ \ \ \ \ \ \ \ \ \ \ {\char'175}}\\
\mbox{\tt >\ {\char'173}\ {\char'45}s\ =\ {\char'136}n{\char'136}>.*\ \ \ \ \ \ \ \ \ \ \ \ \ \ \ \ \ {\char'175}}\\
\mbox{\tt >\ {\char'173}\ {\char'45}c\ =\ {\char'136}n({\char'176}[{\char'136}>{\char'136}w].*|{\char'136}s+{\char'176}{\char'136}w.*)\ {\char'175}}\\
\mbox{\tt }\\[-4pt]
\mbox{\tt >\ \ \ <scrap>\ \ \ ::=\ {\char'45}b{\char'45}s+}\\
\mbox{\tt >\ \ \ <comment>\ ::=\ {\char'45}b{\char'45}c*}\\
\mbox{\tt }\\[-4pt]
\mbox{\tt }\\[-4pt]
\mbox{\tt >{\char'45}{\char'173}}\\
\mbox{\tt >\ scrap\ {\char'137}\ {\char'137}\ inp\ len\ cont\ st\ =\ strip\ len\ inp}\\
\mbox{\tt >\ \ \ \ \ \ \ where}\\
\mbox{\tt >\ \ \ \ \ \ \ strip\ 0\ {\char'137}\ =\ cont\ st}\\
\mbox{\tt >\ \ \ \ \ \ \ strip\ (n+1)\ (c:rst)\ =}\\
\mbox{\tt >\ \ \ \ \ \ \ \ \ \ \ \ \ \ \ if\ c=='{\char'134}n'}\\
\mbox{\tt >\ \ \ \ \ \ \ \ \ \ \ \ \ \ \ \ \ \ then\ '{\char'134}n':strip{\char'137}nl\ n\ rst}\\
\mbox{\tt >\ \ \ \ \ \ \ \ \ \ \ \ \ \ \ \ \ \ else\ c:strip\ n\ rst}\\
\mbox{\tt >}\\
\mbox{\tt >\ \ \ \ \ \ \ strip{\char'137}nl\ (n+1)\ ('>':rst)\ =\ '\ ':strip\ n\ rst}\\
\mbox{\tt >\ \ \ \ \ \ \ strip{\char'137}nl\ n\ rst\ =\ strip\ n\ rst}\\
\mbox{\tt }\\[-4pt]
\mbox{\tt >\ comment\ {\char'137}\ {\char'137}\ inp\ len\ cont\ st\ =\ strip\ len\ inp}\\
\mbox{\tt >\ \ \ \ \ \ \ where}\\
\mbox{\tt >\ \ \ \ \ \ \ strip\ 0\ {\char'137}\ =\ cont\ st}\\
\mbox{\tt >\ \ \ \ \ \ \ strip\ (n+1)\ (c:rst)\ =\ if\ c=='{\char'134}n'\ then\ c:strip\ n\ rst\ else\ strip\ n\ rst}\\
\mbox{\tt }\\[-4pt]
\mbox{\tt }\\[-4pt]
\mbox{\tt >\ literate::\ String\ ->\ String}\\
\mbox{\tt >\ literate\ inp\ =\ drop\ 2\ (gscan\ lit{\char'137}scan\ ()\ ('{\char'134}n':'{\char'134}n':inp))}\\
\mbox{\tt }\\[-4pt]
\mbox{\tt >\ lit{\char'137}scan::\ GScan\ ()\ String}\\
\mbox{\tt >\ lit{\char'137}scan\ =\ load{\char'137}gscan\ (lit{\char'137}acts,stop{\char'137}act)\ lit{\char'137}lx}\\
\mbox{\tt >\ \ \ \ \ \ \ where}\\
\mbox{\tt >\ \ \ \ \ \ \ stop{\char'137}act\ p\ {\char'137}\ ""\ st\ =\ []}\\
\mbox{\tt >\ \ \ \ \ \ \ stop{\char'137}act\ p\ {\char'137}\ {\char'137}\ \ {\char'137}\ \ =\ error\ (msg\ ++\ loc\ p\ ++\ "{\char'134}n")}\\
\mbox{\tt >}\\
\mbox{\tt >\ \ \ \ \ \ \ msg\ \ =\ "literate\ preprocessing\ error\ at\ "}\\
\mbox{\tt >}\\
\mbox{\tt >\ \ \ \ \ \ \ loc\ (Pn\ {\char'137}\ l\ c)\ =\ "line\ "\ ++\ show(l-2)\ ++\ ",\ column\ "\ ++\ show\ c}\\
\mbox{\tt >{\char'45}{\char'175}}
}
\caption{A preprocessor for literate scripts.}\label{fig-lit}
\end{figure}


\subsection{A Simple Preprocessor}
\label{sec-pp}

A general scanner need not return a list at all.  The scanner of
Figure~\ref{fig-pp} is a schematic version of the C preprocessor that only
supports \mbox{\tt {\char'43}include\ "foo"} preprocessor lines.  The return type of the scanner,
and therefore the continuation passed to the action functions, is \mbox{\tt IO\ ()},
which it combines with the \mbox{\tt >>} and \mbox{\tt >>=} operators rather than the \mbox{\tt :} and
\mbox{\tt ++} operators for the list generating scanners.

\begin{figure}
\shrink{
\mbox{\tt {\char'45}{\char'173}}\\
\mbox{\tt import\ Alex}\\
\mbox{\tt {\char'45}{\char'175}}\\
\mbox{\tt }\\[-4pt]
\mbox{\tt }\\[-4pt]
\mbox{\tt "pp{\char'137}lx"/"pp{\char'137}acts":-}\\
\mbox{\tt }\\[-4pt]
\mbox{\tt {\char'173}\ {\char'136}s\ =\ {\char'136}w{\char'43}{\char'136}n\ \ \ \ \ \ \ \ \ \ \ \ \ \ \ \ {\char'175}\ \ \ \ \ \ \ \ \ \ \ --\ spaces\ and\ tabs,\ etc.}\\
\mbox{\tt {\char'173}\ {\char'136}f\ =\ [A-Za-z0-9`{\char'176}{\char'45}-{\char'137}.,/']\ {\char'175}\ \ \ \ \ \ \ \ \ \ \ --\ file-name\ character}\\
\mbox{\tt }\\[-4pt]
\mbox{\tt \ \ <inc>\ ::=\ {\char'136}{\char'43}include{\char'136}s+{\char'136}"{\char'136}f+{\char'136}"{\char'136}s*{\char'136}n}\\
\mbox{\tt \ \ <txt>\ ::=\ .*{\char'136}n}\\
\mbox{\tt }\\[-4pt]
\mbox{\tt }\\[-4pt]
\mbox{\tt {\char'45}{\char'173}}\\
\mbox{\tt inc\ p\ c\ inp\ len\ cont\ st\ =\ pp\ fn\ >>\ cont\ st}\\
\mbox{\tt \ \ \ \ \ \ \ \ where}\\
\mbox{\tt \ \ \ \ \ \ \ \ fn\ =\ (takeWhile\ ('"'/=)\ .\ tail\ .\ dropWhile\ isSpace\ .\ drop\ 8)\ inp}\\
\mbox{\tt }\\[-4pt]
\mbox{\tt txt\ p\ c\ inp\ len\ cont\ st\ =\ putStr\ (take\ len\ inp)\ >>\ cont\ st}\\
\mbox{\tt }\\[-4pt]
\mbox{\tt }\\[-4pt]
\mbox{\tt pp::\ String\ ->\ IO\ ()}\\
\mbox{\tt pp\ fn\ =\ readFile\ fn\ >>=\ {\char'134}cts\ ->\ gscan\ pp{\char'137}scan\ ()\ cts}\\
\mbox{\tt }\\[-4pt]
\mbox{\tt pp{\char'137}scan::\ GScan\ ()\ (IO\ ())}\\
\mbox{\tt pp{\char'137}scan\ =\ load{\char'137}gscan\ (pp{\char'137}acts,stop{\char'137}act)\ pp{\char'137}lx}\\
\mbox{\tt \ \ \ \ \ \ \ \ where}\\
\mbox{\tt \ \ \ \ \ \ \ \ stop{\char'137}act\ {\char'137}\ {\char'137}\ {\char'137}\ {\char'137}\ =\ return\ ()}\\
\mbox{\tt {\char'45}{\char'175}}
}
\caption{A scanner capable of I/O.}\label{fig-pp}
\end{figure}


\subsection{Context Specifications}
\label{sec-context-specifications}

\lx\ retains the facilities of \lex\ for restricting token specifications to
given contexts, albeit in a modified form.  Leading context is specified with
the \mbox{\tt {\char'134}} operator, with its left operand being restricted to a {\em set}
specification.  The character to the left of the token must be matched by the
specification but it is not included in the token.

Trailing context is specified with a \mbox{\tt /}, where the expression to the right can
be a fully-fledged regular expression.  Again, the input to the right of the
token must match the regular expression but it is not included in the token.

Note that unlike \lex, the leading and trailing context do not contribute to
the length of the token when choosing from a number of matching tokens.  The
only effect of the context specifications is to eliminate tokens that would
otherwise match the input.

A token may be restricted to given `start codes' by prefixing the token
specification (to the right of the \mbox{\tt ::=}) with a \mbox{\tt "foo":} where \mbox{\tt foo} is the
name of the start code; to restrict a definition to the \mbox{\tt 0} start code, prefix
it with \mbox{\tt "0":}; if several start codes are given then the scanner may be in any
one of start codes for the token to be selected.
\begin{figure}
\shrink{
\mbox{\tt {\char'45}{\char'173}}\\
\mbox{\tt import\ Alex}\\
\mbox{\tt {\char'45}{\char'175}}\\
\mbox{\tt }\\[-4pt]
\mbox{\tt }\\[-4pt]
\mbox{\tt {\char'173}\ {\char'136}A\ =\ A-Z\ \ \ \ \ \ \ {\char'175}}\\
\mbox{\tt {\char'173}\ {\char'45}t\ =\ [{\char'136}\ {\char'136}t]*{\char'136}n\ {\char'175}}\\
\mbox{\tt }\\[-4pt]
\mbox{\tt "ctx{\char'137}lx"/"ctx{\char'137}acts":-}\\
\mbox{\tt }\\[-4pt]
\mbox{\tt \ \ <only{\char'137}ide>\ \ ::=\ \ \ {\char'136}n{\char'134}{\char'136}A+/{\char'45}t\ \ \ \ \ \ \ \ \ \ \ {\char'45}{\char'173}\ only{\char'137}ide\ \ =\ tkn\ 0\ \ \ \ {\char'45}{\char'175}}\\
\mbox{\tt \ \ <start{\char'137}ide>\ ::=\ \ \ {\char'136}n{\char'134}{\char'136}A+\ \ \ \ \ \ \ \ \ \ \ \ \ \ {\char'45}{\char'173}\ start{\char'137}ide\ =\ tkn\ 1\ \ \ \ {\char'45}{\char'175}}\\
\mbox{\tt \ \ <end{\char'137}ide>\ \ \ ::=\ \ \ \ \ \ {\char'136}A+/{\char'45}t\ \ \ \ \ \ \ \ \ \ \ {\char'45}{\char'173}\ end{\char'137}ide\ \ \ =\ tkn\ 2\ \ \ \ {\char'45}{\char'175}}\\
\mbox{\tt \ \ <ide>\ \ \ \ \ \ \ ::=\ \ \ \ \ \ {\char'136}A+\ \ \ \ \ \ \ \ \ \ \ \ \ \ {\char'45}{\char'173}\ ide\ \ \ \ \ \ \ =\ tkn\ 3\ \ \ \ {\char'45}{\char'175}}\\
\mbox{\tt \ \ <tricky>\ \ \ \ ::=\ \ \ \ \ \ x*/x\ \ \ \ \ \ \ \ \ \ \ \ \ {\char'45}{\char'173}\ tricky\ \ \ \ =\ tkn\ 4\ \ \ \ {\char'45}{\char'175}}\\
\mbox{\tt \ \ <dot>\ \ \ \ \ \ \ ::=\ \ "0":{\char'136}.\ \ \ \ \ \ \ \ \ \ \ \ \ \ \ {\char'45}{\char'173}\ dot\ \ \ \ \ \ \ =\ tkn\ 5\ \ \ \ {\char'45}{\char'175}}\\
\mbox{\tt \ \ <open>\ \ \ \ \ \ ::=\ \ \ \ \ \ {\char'136}(\ \ \ \ \ \ \ \ \ \ \ \ \ \ \ {\char'45}{\char'173}\ open\ \ \ \ \ \ =\ start\ op\ {\char'45}{\char'175}}\\
\mbox{\tt \ \ <comma>\ \ \ \ \ ::=\ "op":{\char'136},\ \ \ \ \ \ \ \ \ \ \ \ \ \ \ {\char'45}{\char'173}\ comma\ \ \ \ \ =\ tkn\ 6\ \ \ \ {\char'45}{\char'175}}\\
\mbox{\tt \ \ <close>\ \ \ \ \ ::=\ \ \ \ \ \ {\char'136})\ \ \ \ \ \ \ \ \ \ \ \ \ \ \ {\char'45}{\char'173}\ close\ \ \ \ \ =\ start\ 0\ \ {\char'45}{\char'175}}\\
\mbox{\tt \ \ <>\ \ \ \ \ \ \ \ \ \ ::=\ \ \ \ \ \ [.{\char'136}n]}\\
\mbox{\tt }\\[-4pt]
\mbox{\tt }\\[-4pt]
\mbox{\tt {\char'45}{\char'173}}\\
\mbox{\tt tkn\ n\ {\char'137}\ {\char'137}\ inp\ len\ cont\ st\ =\ Ide\ n\ (take\ len\ inp):cont\ st}\\
\mbox{\tt }\\[-4pt]
\mbox{\tt start\ sc\ {\char'137}\ {\char'137}\ {\char'137}\ {\char'137}\ cont\ ({\char'137},s)\ =\ cont\ (sc,s)}\\
\mbox{\tt }\\[-4pt]
\mbox{\tt }\\[-4pt]
\mbox{\tt data\ Tkn\ =\ Ide\ Int\ String}\\
\mbox{\tt }\\[-4pt]
\mbox{\tt tokens::\ String\ ->\ [Tkn]}\\
\mbox{\tt tokens\ inp\ =\ gscan\ ctx{\char'137}scan\ ()\ inp}\\
\mbox{\tt }\\[-4pt]
\mbox{\tt ctx{\char'137}scan::\ GScan\ ()\ [Tkn]}\\
\mbox{\tt ctx{\char'137}scan\ =\ load{\char'137}gscan\ (ctx{\char'137}acts,stop{\char'137}act)\ ctx{\char'137}lx}\\
\mbox{\tt \ \ \ \ \ \ \ \ where}\\
\mbox{\tt \ \ \ \ \ \ \ \ stop{\char'137}act\ {\char'137}\ {\char'137}\ ""\ {\char'137}\ =\ []}\\
\mbox{\tt \ \ \ \ \ \ \ \ stop{\char'137}act\ {\char'137}\ {\char'137}\ {\char'137}\ \ {\char'137}\ =\ error\ "tokens"}\\
\mbox{\tt {\char'45}{\char'175}}
}
\caption{Specifying context in \lx.}\label{fig-ctx}
\end{figure}

Each start code, \mbox{\tt "foo"}, mentioned in the script will result in a definition
like

\display{
\mbox{\tt foo\ =\ 1}
}

being added to the output (where the integer is positive and distinct from
those assigned to other start codes) so each start code must be a Haskell
function identifier that is not otherwise bound in the output module.

By default, the scanner starts in start code \mbox{\tt 0}.  To change to a different
start code, an action function has only to call its continuation function with
the \mbox{\tt StartCode} component of its state argument set to a different value; this
will immediately eliminate token definitions annotated with start codes that do
not include the new start code.

Figure~\ref{fig-ctx} gives a rather contrived scanner that illustrates several
aspects of context specifications.

The first four tokens specifications match the same text, only differing in
their context.  \mbox{\tt <only{\char'137}ide>} will be selected if it is the only alphabetic
string on the line, \mbox{\tt <start{\char'137}ide>} if it is at the start of the line,
\mbox{\tt <end{\char'137}ide>} if it is at the end of a line, otherwise \mbox{\tt <ide>}.  Note that there
is no question of the leading or trailing context elongating any of the tokens
and interfering with their priority.

The \mbox{\tt <tricky>} definition would not work properly in \lex, the final \mbox{\tt x} in the
specification being included in the token.  \lx\ does not suffer from this
problem.

The \mbox{\tt <dot>} token is restricted to start code 0 so dot characters will be
initially recognised, but they will be ignored after a \mbox{\tt (} token has been read,
as it changes the start code to \mbox{\tt op}.  Once a \mbox{\tt )} has been read, the start code
will revert to 0 and the dots will resume.  On the other hand commas are
restricted to the \mbox{\tt op} start code so they will be initially disabled, appearing
between open and close brackets.


\section*{Acknowledgements}

I would like to thank Tom Alardice for providing the original motivation for
writing \lx, Henk Muller for suggestions and encouragement, Ian Holyer for the
various discussions that have helped to shape it and Alastair Reid for feedback
and suggestions.
\end{document}
